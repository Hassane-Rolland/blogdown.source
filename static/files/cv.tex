\documentclass[10pt,]{article}
\usepackage[sc, osf]{mathpazo}
\usepackage{amssymb,amsmath}
\usepackage{ifxetex,ifluatex}
\usepackage{fixltx2e} % provides \textsubscript
\ifnum 0\ifxetex 1\fi\ifluatex 1\fi=0 % if pdftex
  \usepackage[T1]{fontenc}
  \usepackage[utf8]{inputenc}
\else % if luatex or xelatex
  \ifxetex
    \usepackage{mathspec}
  \else
    \usepackage{fontspec}
  \fi
  \defaultfontfeatures{Ligatures=TeX,Scale=MatchLowercase}
\fi
% use upquote if available, for straight quotes in verbatim environments
\IfFileExists{upquote.sty}{\usepackage{upquote}}{}
% use microtype if available
\IfFileExists{microtype.sty}{%
\usepackage{microtype}
\UseMicrotypeSet[protrusion]{basicmath} % disable protrusion for tt fonts
}{}
\usepackage[margin=1in]{geometry}




\setlength{\emergencystretch}{3em}  % prevent overfull lines
\providecommand{\tightlist}{%
  \setlength{\itemsep}{0pt}\setlength{\parskip}{0pt}}
\setcounter{secnumdepth}{0}
% Redefines (sub)paragraphs to behave more like sections
\ifx\paragraph\undefined\else
\let\oldparagraph\paragraph
\renewcommand{\paragraph}[1]{\oldparagraph{#1}\mbox{}}
\fi
\ifx\subparagraph\undefined\else
\let\oldsubparagraph\subparagraph
\renewcommand{\subparagraph}[1]{\oldsubparagraph{#1}\mbox{}}
\fi

% Now begins the stuff that I added.
% ----------------------------------

% Custom section fonts
\usepackage{sectsty}
\sectionfont{\rmfamily\mdseries\large\bf}
\subsectionfont{\rmfamily\mdseries\normalsize\itshape}


% Make lists without bullets
%\renewenvironment{itemize}{
%  \begin{list}{}{
%    \setlength{\leftmargin}{1.5em}
%  }
%}{
%  \end{list}
%}


% Make parskips rather than indent with lists.
\usepackage{parskip}
\usepackage{titlesec}
\titlespacing\section{0pt}{12pt plus 4pt minus 2pt}{4pt plus 2pt minus 2pt}
\titlespacing\subsection{0pt}{12pt plus 4pt minus 2pt}{4pt plus 2pt minus 2pt}

% Use fontawesome. Note: you'll need TeXLive 2015. Update.


% Fancyhdr, as I tend to do with these personal documents.
\usepackage{fancyhdr,lastpage}
\pagestyle{fancy}
\renewcommand{\headrulewidth}{0.0pt}
\renewcommand{\footrulewidth}{0.0pt}
\lhead{}
\chead{}
\rhead{}
\lfoot{
\cfoot{\scriptsize  Hassane Rolland - CV -  }}
\rfoot{\scriptsize \thepage/{\hypersetup{linkcolor=black}\pageref{LastPage}}}

% Always load hyperref last.
\usepackage{hyperref}
\PassOptionsToPackage{usenames,dvipsnames}{color} % color is loaded by hyperref

\hypersetup{unicode=true,
            pdftitle={Hassane Rolland:  CV (Curriculum Vitae)},
            pdfauthor={Hassane Rolland},
            pdfkeywords={Biologie, Science des données},
            colorlinks=true,
            linkcolor=blue,
            citecolor=Blue,
            urlcolor=blue,
            breaklinks=true, bookmarks=true}
\urlstyle{same}  % don't use monospace font for urls

\begin{document}


\centerline{\huge \bf Hassane Rolland}

\vspace{2 mm}

\hrule

\vspace{2 mm}

\moveleft.5\hoffset\centerline{Etudiant à l'Université de Mons}
\moveleft.5\hoffset\centerline{8 avenue du Champ de Mars, 7000 Mons, Belgique}
\moveleft.5\hoffset\centerline{ \emph{E-mail:} \href{mailto:}{\tt \href{mailto:Hassane.RollandE@student.umons.ac.be}{\nolinkurl{Hassane.RollandE@student.umons.ac.be}}} \hspace{1 mm}  \emph{Github:} \href{http://github.com/Hassane-Rolland}{\tt Hassane-Rolland} \hspace{1 mm}      | \emph{Updated:} \today}

\vspace{2 mm}

\hrule


\section{EDUCATION}\label{education}

\emph{Université de Mons}, Bachelier en Biologie \hfill 2020

\emph{Ecole secondaire}, orientation \hfill 2017

\section{EMPLOIS}\label{emplois}

\emph{Job étudiant:}

\begin{quote}
Activité réalisée \hfill 2017--2018
\end{quote}

\emph{Autre emploi:}

\begin{quote}
Rôle 2 \hfill 2016--2018
\end{quote}

\begin{quote}
Rôle 1 \hfill 2015
\end{quote}

\section{PUBLICATIONS}\label{publications}

\subsection{\texorpdfstring{\textbf{Articles
scientifiques}}{Articles scientifiques}}\label{articles-scientifiques}

\ldots{}

\subsection{\texorpdfstring{\textbf{Livres}}{Livres}}\label{livres}

\ldots{}

\section{RECOMPENSES}\label{recompenses}

\textbf{2017} \emph{Prix XXX}, détails concernant ce prix

\end{document}
